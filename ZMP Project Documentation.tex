\documentclass{article}
\usepackage{polski}
\usepackage[utf8]{inputenc}
\usepackage{hyperref}
\usepackage{natbib}
\usepackage{graphicx}
\graphicspath{ {./images/} }
\usepackage[rightcaption]{sidecap}
\usepackage{wrapfig}

\hypersetup{
    colorlinks=true,
    linkcolor=blue,
    filecolor=magenta,      
    urlcolor=cyan,
    pdfpagemode=FullScreen,
    }

\title{ZMP\textunderscore Obserwowator cen produktów w sklepie  internetowym STEAM}

\author{
Piotr Tekieli,\\
Jan Kwiatkowski,\\
Mariusz Skuza,\\
Szymon Bacański
}

\date{Kwiecień 2022}

\begin{document}
\maketitle

\section{Opis funkcjonalny systemu}
Serwis internetowy zostanie wykonany w oparciu o specyfikację projektową. Projektowany od początku serwis musi być elastyczny, tj. powinien umożliwić rozwój istniejących i dodawanie nowych funkcjonalności bez konieczności przebudowy znacznych części kodu lub architektury.
\\
Celem projektu jest stworzenie aplikację do obserwowania cen produktów w sklepie internetowym STEAM. Serwis będzie się składał z aplikacji WEB, Mobile, Desktop, które będą łączyć się do stworzonym API.

    \subsection{Opis funkcjonalny API}
        Api będze pełniło funkcję głównego węzła komunikacyjnego dla systemu poprzez:
        \begin{enumerate}
            \item Wysyłanie i odbieranie informacji do innych aplikacji odnoście uzytkowników jak i przypisanych do konta informacji
            \item Prowadzenie bazy danych z informacjami odnośnie kont i gier
            \item Aktualizacja danych z sklepu STEAM 
        \end{enumerate}
        API ma w sposób dostępny i szybki umożliwić innym aplikacą na dostep do informacji wiązanymi z treścią serwisu. 
        
    \subsection{Opis funkcjonalny Web}
         Witryna internetowa wyposażona w narzędzia ułatwiające nawigację i orientację w zawartości serwisu. Główną jej funkcją będze kontakt z użytkownikam czyli umożliwienie dostepu do informacji
         serwisu i czytelne przedstawienie ich.
         \\
         Według wstępnych założeń witryna zawierać będzie:
        \begin{enumerate}
            \item Strona logowania/rejestracji
            \item Wyszukiwarkę
            \item Odnośnik do dodania gry do obserwowanych
            \item Wgląd do historji ceny obserwowanego produktu
        \end{enumerate}
        Dopuszcza się modyfikację tej struktury na etapie budowy innych elemętów serwisu.
        
    \subsection{Opis funkcjonalny Desktop}
        Aplikacja na komputer pozwalajaca kożystać z serisu.\\
        Według wstępnych założeń Aplikacja zawierać będzie:
        \begin{enumerate}
            \item Strona logowania/rejestracji
            \item Wyszukiwarkę
            \item Dodanie gry do obserwowanych
            \item Usuwanie z obserwacji
            \item Wgląd do historji ceny obserwowanego produktu porzez diagram liniowy
            \item Automatyczne logowanie
        \end{enumerate}
        Dopuszcza się modyfikację tej struktury na etapie budowy innych elemętów serwisu.
        
    \subsection{Opis funkcjonalny Mobile}
        Aplikacja mobilna na androida pozwalajaca kożystać z serisu.\\
        Według wstępnych założeń Aplikacja zawierać będzie:
            \begin{enumerate}
            \item Strona logowania/rejestracji
            \item Wyszukiwarkę
            \item Dodanie gry do obserwowanych
            \item Usuwanie z obserwacji
            \item Wgląd do historji ceny obserwowanego produktu porzez diagram liniowy
            \item Automatyczne logowanie
        \end{enumerate}
        Dopuszcza się modyfikację tej struktury na etapie budowy innych elemętów serwisu.

\section{Wykorzystywane technologie}
    \subsection{API}
    \begin{itemize}
            \item Node.js\\
            Wieloplatformowe oppensorce środowisko do tworzenia aplikacji typu server-side napisanych w języku JavaScript.
            \item MongoDB\\
             Otwarty, nierelacyjny system zarządzania bazą danych napisany w języku C++. Charakteryzuje się brakiem ściśle zdefiniowanej struktury obsługiwanych baz danych. Zamiast tego dane składowane są jako dokumenty w stylu JSON.
             \end{itemize}
             
    \subsection{Web}
    \begin{itemize}
            \item React.js\\
            Biblioteka języka programowania JavaScript, która wykorzystywana jest do tworzenia interfejsów graficznych aplikacji internetowych.Zainspirowana przez rozszerzenie języka PHP – XHP. Często wykorzystywana do tworzenia aplikacji typu Single Page Application
             \end{itemize}
        
    \subsection{Desktop}
    \begin{itemize}
            \item Electron\\
            Otwartoźródłowa platforma programistyczna pozwaljąca tworzyć aplikacje GUI dla komputerów stacjonarnych za pomocą komponentów elementów front-endowych i back-endowych, opracowanych początkowo dla aplikacji sieciowych: Node.js (back-end) i Chromium (front-end). Electron jest główną strukturą GUI za kilkoma znaczącymi projektami open source, w tym edytorami kodu źródłowego Atom i Visual Studio Code oraz czatem Discord.
            \item Chromium\\
            Otwarty projekt przeglądarki internetowej, z którego kod źródłowy czerpią między innymi Google Chrome, Opera czy Microsoft Edge.
            \item Node.js\\
            Wieloplatformowe oppensorce środowisko do tworzenia aplikacji typu server-side napisanych w języku JavaScript.
             \end{itemize}
             
    \subsection{Mobile}
     \begin{itemize}
            \item Xamarin.Forms\\
             To wieloplatformowa abstrakcyjna platforma narzędziowa z interfejsem użytkownika, która umożliwia programistom łatwe tworzenie interfejsów użytkownika, które można udostępniać na urządzeniach z Androidem, iOS, Windows i Windows Phone.
            \item Xamarin.Community\\
            Zestaw narzędzi społeczności Xamarin to kolekcja animacji, zachowań, konwerterów i efektów do tworzenia aplikacji mobilnych przy użyciu platformy Xamarin.Forms. Upraszcza i demonstruje typowe zadania deweloperskie.
             \end{itemize}


\section{Wzorce projektowe}
    \subsection{API}
       
        
    \subsection{Web}
        
        
    \subsection{Desktop}
    \begin{itemize}
           \item Budowniczy\\
             Budowniczy jest kreacyjnym wzorcem projektowym, który daje możliwość tworzenia złożonych obiektów etapami, krok po kroku. Wzorzec ten pozwala produkować różne typy oraz reprezentacje obiektu używając tego samego kodu konstrukcyjnego.
             \end{itemize}
    \subsection{Mobile}
    \begin{itemize}
           \item MVVM\\
             Wzorzec Model-View-ViewModel - opiera się na wydzieleniu odpowiednich warstw w systemie, w celu podziału zadań oraz zmniejszenia zależności pomiędzy klasami. Mamy więc klasy modelu danych, których zadaniem jest przechowywanie danych właśnie oraz ich ewentualną walidację.
             \end{itemize}

\section{Instrukcję lokalnego i zdalnego uruchomienia systemu}
    \subsection{API}
        \subsubsection{Postawienie aplikacji lokalnie}
        \begin{enumerate}
            \item Clone the github repository\\
                        \emph{git clone https://github.com/Price-Tracker-ZMP/API/} \\
            \item Open the project and install NPM\\
              \emph{npm install} \\
            \item Run the application with:\\
                \emph{npm start} \\
        \end{enumerate}
        
        \subsubsection{Postawienie aplikacji zdalnie}
        
        \subsubsection{Instrukcja uruchomienia testów}
        
    \subsection{Web}
        \subsubsection{Postawienie aplikacji lokalnie}
        \begin{enumerate}
            \item Clone the github repository\\
                        \emph{git clone https://github.com/Price-Tracker-ZMP/Web/} \\
            \item Open the project and install NPM\\
              \emph{npm install} \\
            \item Run the application with:\\
                \emph{npm start} \\
        \end{enumerate}
        
        \subsubsection{Postawienie aplikacji zdalnie}
        
        \subsubsection{Instrukcja uruchomienia testów}
        
    \subsection{Desktop}
        \subsubsection{Postawienie aplikacji deweloper}
        \begin{enumerate}
            \item Clone the github repository\\
                        \emph{git clone https://github.com/Price-Tracker-ZMP/Desktop/} \\
            \item Open the project and install NPM\\
              \emph{npm install} \\
            \item Run the application with:\\
                \emph{npm start} \\
        \end{enumerate}
   
        \subsubsection{Postawienie aplikacji użytkownik}
        
        \subsubsection{Instrukcja uruchomienia testów}
        
    \subsection{Mobile}
        \subsubsection{Postawienie aplikacji deweloper}
         \begin{enumerate}
            \item Install Visual Studio\\
            \item Install Visual Studio Extension for Xamarin\\
                \includegraphics{mobile_step_2}\\
            \item Clone repository\\
            \href{https://github.com/Price-Tracker-ZMP/Mobile.git}{Git repository}\\
            \item Open project (install additionals if neccessary)\\
                \includegraphics{mobile_step_4}\\
            \item Start project (during first start, install Android emulator and default device)\\
                \includegraphics{mobile_step_5}\\
        \end{enumerate}
       
        \subsubsection{Postawienie aplikacji użytkownik}
        
        \subsubsection{Instrukcja uruchomienia testów}

\section{Schematy i diagramy}

\section{Repozytorium}
Link do naszej organizacji: \\
\url{https://github.com/Price-Tracker-ZMP}\\\\

\section{Wnioski}

\end{document}
