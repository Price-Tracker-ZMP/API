\documentclass{article}
\usepackage{polski}
\usepackage[utf8]{inputenc}
\usepackage{hyperref}
\usepackage{natbib}
\usepackage{graphicx}
\graphicspath{ {./images/} }
\usepackage[rightcaption]{sidecap}
\usepackage{wrapfig}

\hypersetup{
    colorlinks=true,
    linkcolor=blue,
    filecolor=magenta,      
    urlcolor=cyan,
    pdfpagemode=FullScreen,
    }

\title{ZMP\textunderscore Obserwowator cen produktów w sklepie  internetowym STEAM}

\author{
Piotr Tekieli,\\
Jan Kwiatkowski,\\
Mariusz Skuza,\\
Szymon Bacański
}

\date{Kwiecień 2022}

\begin{document}
\maketitle

\section{Opis funkcjonalny systemu}
Serwis internetowy zostanie wykonany w oparciu o specyfikację projektową. Projektowany od początku serwis musi być elastyczny, tj. powinien umożliwić rozwój istniejących i dodawanie nowych funkcjonalności bez konieczności przebudowy znacznych części kodu lub architektury.
\\
Celem projektu jest stworzenie aplikację do obserwowania cen produktów w sklepie internetowym STEAM. Serwis będzie się składał z aplikacji WEB, Mobile, Desktop, które będą łączyć się do stworzonym API.

    \subsection{Opis funkcjonalny API}
        Api będze pełniło funkcję głównego węzła komunikacyjnego dla systemu poprzez:
        \begin{enumerate}
            \item Wysyłanie i odbieranie informacji do innych aplikacji odnoście uzytkowników jak i przypisanych do konta informacji
            \item Prowadzenie bazy danych z informacjami odnośnie kont i gier
            \item Aktualizacja danych z sklepu STEAM 
        \end{enumerate}
        API ma w sposób dostępny i szybki umożliwić innym aplikacą na dostep do informacji wiązanymi z treścią serwisu. 
        
    \subsection{Opis funkcjonalny Web}
         Witryna internetowa wyposażona w narzędzia ułatwiające nawigację i orientację w zawartości serwisu. Główną jej funkcją będze kontakt z użytkownikam czyli umożliwienie dostepu do informacji
         serwisu i czytelne przedstawienie ich.
         \\
         Według wstępnych założeń witryna zawierać będzie:
        \begin{enumerate}
            \item Strona logowania/rejestracji
            \item Wyszukiwarkę
            \item Odnośnik do dodania gry do obserwowanych
            \item Wgląd do historji ceny obserwowanego produktu
        \end{enumerate}
        Dopuszcza się modyfikację tej struktury na etapie budowy innych elemętów serwisu.
        
    \subsection{Opis funkcjonalny Desktop}
        Aplikacja na komputer pozwalajaca kożystać z serisu.\\
        Według wstępnych założeń Aplikacja zawierać będzie:
        \begin{enumerate}
            \item Strona logowania/rejestracji
            \item Wyszukiwarkę
            \item Dodanie gry do obserwowanych
            \item Usuwanie z obserwacji
            \item Wgląd do historji ceny obserwowanego produktu porzez diagram liniowy
            \item Automatyczne logowanie
        \end{enumerate}
        Dopuszcza się modyfikację tej struktury na etapie budowy innych elemętów serwisu.
        
    \subsection{Opis funkcjonalny Mobile}
        Aplikacja mobilna na androida pozwalajaca kożystać z serisu.\\
        Według wstępnych założeń Aplikacja zawierać będzie:
            \begin{enumerate}
            \item Strona logowania/rejestracji
            \item Wyszukiwarkę
            \item Dodanie gry do obserwowanych
            \item Usuwanie z obserwacji
            \item Wgląd do historji ceny obserwowanego produktu porzez diagram liniowy
            \item Automatyczne logowanie
        \end{enumerate}
        Dopuszcza się modyfikację tej struktury na etapie budowy innych elemętów serwisu.

\section{Wykorzystywane technologie}
    \subsection{API}
    \begin{itemize}
            \item Node.js\\
            Wieloplatformowe oppensorce środowisko do tworzenia aplikacji typu server-side napisanych w języku JavaScript.
            \item MongoDB\\
             Otwarty, nierelacyjny system zarządzania bazą danych napisany w języku C++. Charakteryzuje się brakiem ściśle zdefiniowanej struktury obsługiwanych baz danych. Zamiast tego dane składowane są jako dokumenty w stylu JSON.
             \end{itemize}
             
    \subsection{Web}
    \begin{itemize}
            \item React.js\\
            Biblioteka języka programowania JavaScript, która wykorzystywana jest do tworzenia interfejsów graficznych aplikacji internetowych.Zainspirowana przez rozszerzenie języka PHP – XHP. Często wykorzystywana do tworzenia aplikacji typu Single Page Application
             \end{itemize}
        
    \subsection{Desktop}
    \begin{itemize}
            \item Electron\\
            Otwartoźródłowa platforma programistyczna pozwaljąca tworzyć aplikacje GUI dla komputerów stacjonarnych za pomocą komponentów elementów front-endowych i back-endowych, opracowanych początkowo dla aplikacji sieciowych: Node.js (back-end) i Chromium (front-end). Electron jest główną strukturą GUI za kilkoma znaczącymi projektami open source, w tym edytorami kodu źródłowego Atom i Visual Studio Code oraz czatem Discord.
            \item Chromium\\
            Otwarty projekt przeglądarki internetowej, z którego kod źródłowy czerpią między innymi Google Chrome, Opera czy Microsoft Edge.
            \item Node.js\\
            Wieloplatformowe oppensorce środowisko do tworzenia aplikacji typu server-side napisanych w języku JavaScript.
             \end{itemize}
             
    \subsection{Mobile}
     \begin{itemize}
            \item Xamarin.Forms\\
             To wieloplatformowa abstrakcyjna platforma narzędziowa z interfejsem użytkownika, która umożliwia programistom łatwe tworzenie interfejsów użytkownika, które można udostępniać na urządzeniach z Androidem, iOS, Windows i Windows Phone.
            \item Xamarin.Community\\
            Zestaw narzędzi społeczności Xamarin to kolekcja animacji, zachowań, konwerterów i efektów do tworzenia aplikacji mobilnych przy użyciu platformy Xamarin.Forms. Upraszcza i demonstruje typowe zadania deweloperskie.
             \end{itemize}


\section{Wzorce projektowe}
    \subsection{API}
       \begin{itemize}
           \item Circuit Breaker\\
             Wzorzec Circuit Breaker, zapewnia sposób na wykrycie popsutej zależności i zatrzymuje przepływ danych i pozwala uniknąć opóźnień i okropnego UX.
Wykozystujemy ten wzorzec do komunikacji z serwisami wewnętrznymi.
            \end{itemize}   
    \subsection{Web}
  \begin{itemize}
           \item Singleton\\
             Singleton jest wzorcem, który pozwala na stworzenie tylko jednej instancji obiektu z klasy bądź konstruktora funkcyjnego.  W przypadku wielokrotnego wywoływania tej samej klasy zawsze będziemy otrzymywali tą samą instancję, która została stworzona podczas pierwszego wywołania. 
            \end{itemize}     
    \subsection{Desktop}
    \begin{itemize}
           \item Budowniczy\\
             Budowniczy jest kreacyjnym wzorcem projektowym, który daje możliwość tworzenia złożonych obiektów etapami, krok po kroku. Wzorzec ten pozwala produkować różne typy oraz reprezentacje obiektu używając tego samego kodu konstrukcyjnego.
             \end{itemize}
    \subsection{Mobile}
    \begin{itemize}
           \item MVVM\\
             Wzorzec Model-View-ViewModel - opiera się na wydzieleniu odpowiednich warstw w systemie, w celu podziału zadań oraz zmniejszenia zależności pomiędzy klasami. Mamy więc klasy modelu danych, których zadaniem jest przechowywanie danych właśnie oraz ich ewentualną walidację.
             \end{itemize}

\section{Instrukcję lokalnego i zdalnego uruchomienia systemu}
    \subsection{API}
        \subsubsection{Postawienie aplikacji lokalnie}
        \begin{enumerate}
            \item Clone the github repository\\
                        \emph{git clone https://github.com/Price-Tracker-ZMP/API/} \\
            \item Open the project and install NPM\\
              \emph{npm install} \\
            \item Run the application with:\\
                \emph{npm start} \\
        \end{enumerate}
        
        \subsubsection{Postawienie aplikacji zdalnie}
        Na maszynie (serwerze) z zainstalowanymi pakietami:\\
        \begin{enumerate}
            \item nodejs\\
            \item pm2\\
            \item git\\
            \item nginx\\
        \end{enumerate}
        Z pomocą połącznienia ssh.\\
        
        \begin{enumerate}
            \item Clone the github repository\\
                \emph{git clone git@github.com:Price-Tracker-ZMP/API.git} \\
            \item Open the project and install NPM\\
              \emph{npm install} \\
            \item Run the application with:\\
                \emph{pm2 start api.js} \\
            \item Change port on file:\\
                \emph{sudo nano /etc/nginx/sides-enabled/default} \\
            \item Reload file\\
                \emph{sudo /etc/init.d/nginx reload} \\
        \end{enumerate}
        
        \subsubsection{Instrukcja uruchomienia testów}
         \begin{enumerate}
            \item Testy uruchamia w terminalu\\
            \emph{npm run tests} \\
        \end{enumerate}
    \subsection{Web}
        \subsubsection{Postawienie aplikacji lokalnie}
        \begin{enumerate}
            \item Clone the github repository\\
                        \emph{git clone https://github.com/Price-Tracker-ZMP/Web/} \\
            \item Open the project and install NPM\\
              \emph{npm install} \\
            \item Run the application with:\\
                \emph{npm start} \\
        \end{enumerate}
        
        \subsubsection{Postawienie aplikacji zdalnie}
         \begin{enumerate}
            \item Wchodzimy na strone vercel.com\\
            \item Logujemy się z pomoca github\\
            \item Importujemy dane repozytorium z aplikacją nazwyając ją\\
            \item Ustawiamy framework na react app i klikamy deploy\\
            \item Czekamy za zbudowanie\\
            \item Strona dostępna w internecie dod domeną "nazwa-projektu".vercel.com\\
        \end{enumerate}
        
        
        \subsubsection{Instrukcja uruchomienia testów}
         \begin{enumerate}
            \item Testy uruchamia w terminalu\\
            \emph{npm run tests} \\
        \end{enumerate}
    \subsection{Desktop}
        \subsubsection{Postawienie aplikacji deweloper}
        \begin{enumerate}
            \item Clone the github repository\\
                        \emph{git clone https://github.com/Price-Tracker-ZMP/Desktop/} \\
            \item Open the project and install NPM\\
              \emph{npm install} \\
            \item Run the application with:\\
                \emph{npm start} \\
        \end{enumerate}
   
        \subsubsection{Postawienie aplikacji użytkownik}
        \begin{enumerate}
            \item Kliknij drukrotnie w instalkę\\
                \includegraphics{ikona desktop.PNG}\\
            \item Animacja instalacji\\
                \includegraphics{CNQB8fV.png}\\
            \item Aplikacja gotowa do użytku\\
        \end{enumerate}
        
        \subsubsection{Instrukcja uruchomienia testów}
        \begin{enumerate}
            \item Testy uruchamia w terminalu\\
            \emph{npm run tests} \\
        \end{enumerate}
        
    \subsection{Mobile}
        \subsubsection{Postawienie aplikacji deweloper}
         \begin{enumerate}
            \item Install Visual Studio\\
            \item Install Visual Studio Extension for Xamarin\\
                \includegraphics{mobile_step_2}\\
            \item Clone repository\\
            \href{https://github.com/Price-Tracker-ZMP/Mobile.git}{Git repository}\\
            \item Open project (install additionals if neccessary)\\
                \includegraphics{mobile_step_4}\\
            \item Start project (during first start, install Android emulator and default device)\\
                \includegraphics{mobile_step_5}\\
        \end{enumerate}
       
        \subsubsection{Postawienie aplikacji użytkownik}
            \begin{enumerate}
            \item  Wejdź na sklep play.\\
            \item Znajdź aplikację pod nazwą PriceTracker lub wejdź w link\\ \href{https://play.google.com/store/apps/details?id=com.merfeusz.pricetrackermobile}{https://play.google.com/store/apps/details?id=com.merfeusz.pricetrackermobile}\\
            \item  Kliknij zainstaluj\\
        \end{enumerate}
      
        \subsubsection{Instrukcja uruchomienia testów}
             \begin{enumerate}
                \item Dodaj zmienne środowiskowe (ANDROID HOME i JAVA HOME)" potrzebne do uruchomienia testów.\\
                    \includegraphics{images/1unknown.png}\\
                \item Zainstaluj NDK.\\
                    \includegraphics{images/2unknown.png}\\
                \item Skonfiguruj package name aplikacji dla inicjatora.\\
                    \includegraphics{images/3unknown.png}\\
                \item Uruchom projekt na emulatorze, aby mieć pewność, że aplikacja jest zainstalowana.\\
                \item Uruchom testy w zakładce Test Explorer.\\
                Propane zdane testy\\
                    \includegraphics{images/5unknown.png}\\
        \end{enumerate}

\section{Schematy i diagramy}
    \includegraphics{images/baza danych.PNG}\\
    
\section{Repozytorium}
Link do naszej organizacji: \\
\url{https://github.com/Price-Tracker-ZMP}\\\\

\section{Wnioski}
   Projekt ten nie był aż tak prostym przedsięwzięciem jak zakładaliśmy na starcie. Jednak ucząc się po drodze udało się nam go ukończyć. Nigdy wcześniej część naszego zespołu nie korzystała z jawa skryptu i innych wyczytanych technologii, co także utrudniło pracę. Więc musieliśmy pomagać sobie, a najbardziej osobom, które nie były zaznajomione z ów językiem, aby poprawnie utworzyć naszą aplikację.\\
    Poza techniczną stroną projektu musieliśmy się też zmierzyć z trudność zarządzania czasem z powodu natłoku innych sytuacji losowych i pracy zawodowej.\\
    Ale miło to tez dobą stronę, ponieważ niektórzy członkowie naszego zespołu mieli już styczność z wykorzystywanymi technologiami w życiu zawodowym.\\
    Musiała to być praca wspólna, aby osiągnąć nasz cel końcowy.\\
    
    Wynieśliśmy z tego następujące wnioski:\\
    \begin{itemize}
        \item Przygotowanie aplikacji wymaga wiedzy poza czystą wiedzą techniczną.
        \item Warto posiadać członków zespołu, którzy posiadają głęboką wiedzę w danym zakresie.
        \item Organizacja grupy może być wymagająca, gdy niektórzy członkowie są osobami pracującymi.
        \item Flow pracy może łatwo być przerwany w wypadku chorób i różnych prywatnych spraw w zespole.
        \item Nie należy być zbyt optymistycznym w przypadku planowania czasu pracy.
    \end{itemize}


\end{document}
