\documentclass{article}
\usepackage{polski}
\usepackage[utf8]{inputenc}
\usepackage{hyperref}
\usepackage{natbib}
\usepackage{graphicx}

\hypersetup{
    colorlinks=true,
    linkcolor=blue,
    filecolor=magenta,      
    urlcolor=cyan,
    pdfpagemode=FullScreen,
    }

\title{ZMP\textunderscore Obserwowator cen produktów w sklepie  internetowym STEAM}

\author{
Piotr Tekieli,\\
Jan Kwiatkowski,\\
Mariusz Skuza,\\
Szymon Bacański
}

\date{Kwiecień 2022}

\begin{document}
\maketitle

\section{Opis funkcjonalny systemu}
Serwis internetowy zostanie wykonany w oparciu o specyfikację projektową. Projektowany od początku serwis musi być elastyczny, tj. powinien umożliwić rozwój istniejących i dodawanie nowych funkcjonalności bez konieczności przebudowy znacznych części kodu lub architektury.
\\
Celem projektu jest stworzenie aplikację do obserwowania cen produktów w sklepie internetowym STEAM. Serwis będzie się składał z aplikacji WEB, Mobile, Desktop, które będą łączyć się do stworzonym API.

    \subsection{Opis funkcjonalny API}
        Api będze pełniło funkcję głównego węzła komunikacyjnego dla systemu poprzez:
        \begin{enumerate}
            \item Wysyłanie i odbieranie informacji do innych aplikacji odnoście uzytkowników jak i przypisanych do konta informacji
            \item Prowadzenie bazy danych z informacjami odnośnie kont i gier
            \item Aktualizacja danych z sklepu STEAM 
        \end{enumerate}
        API ma w sposób dostępny i szybki umożliwić innym aplikacą na dostep do informacji wiązanymi z treścią serwisu. 
        
    \subsection{Opis funkcjonalny Web}
         Witryna internetowa wyposażona w narzędzia ułatwiające nawigację i orientację w zawartości serwisu. Główną jej funkcją będze kontakt z użytkownikam czyli umożliwienie dostepu do informacji
         serwisu i czytelne przedstawienie ich.
         \\
         Według wstępnych założeń witryna zawierać będzie:
        \begin{enumerate}
            \item Strona logowania/rejestracji
            \item Wyszukiwarkę
            \item Odnośnik do dodania gry do obserwowanych
            \item Wgląd do historji ceny obserwowanego produktu
        \end{enumerate}
        
        Dopuszcza się modyfikację tej struktury na etapie budowy innych elemętów serwisu.
        
    \subsection{Opis funkcjonalny Desktop}
    
        \begin{enumerate}
            \item Strona logowania/rejestracji
            \item Wyszukiwarkę
            \item Dodanie gry do obserwowanych
            \item Usuwanie z obserwacji
            \item Wgląd do historji ceny obserwowanego produktu porzez diagram liniowy
            \item Automatyczne logowanie
        \end{enumerate}
    \subsection{Opis funkcjonalny Mobile}
            \begin{enumerate}
            \item Strona logowania/rejestracji
            \item Wyszukiwarkę
            \item Dodanie gry do obserwowanych
            \item Usuwanie z obserwacji
            \item Wgląd do historji ceny obserwowanego produktu porzez diagram liniowy
            \item Automatyczne logowanie
        \end{enumerate}

\section{Streszczenie opisu technologicznego dla}
    \subsection{API}
       
        
    \subsection{Web}
        
        
    \subsection{Desktop}
    
    \subsection{Mobile}
    

\section{Streszczenie wzorców projektowych}
    \subsection{API}
       
        
    \subsection{Web}
        
        
    \subsection{Desktop}
    
    \subsection{Mobile}




\section{Instrukcję lokalnego i zdalnego uruchomienia systemu}
    \subsection{API}
        \subsubsection{Postawienie aplikacji lokalnie}
        \subsubsection{Postawienie aplikacji zdalnie}
        \subsubsection{Instrukcja uruchomienia testów}
    \subsection{Web}
        \subsubsection{Postawienie aplikacji lokalnie}
        \subsubsection{Postawienie aplikacji zdalnie}
        \subsubsection{Instrukcja uruchomienia testów}
    \subsection{Desktop}
        \subsubsection{Postawienie aplikacji lokalnie}
        \subsubsection{Postawienie aplikacji zdalnie}
        \subsubsection{Instrukcja uruchomienia testów}
    \subsection{Mobile}
        \subsubsection{Postawienie aplikacji lokalnie}
        \subsubsection{Postawienie aplikacji zdalnie}
        \subsubsection{Instrukcja uruchomienia testów}



\end{document}